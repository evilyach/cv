% Left frame
%%%%%%%%%%%%%%%%%%%%
\begin{figure}
	\hfill
	\includegraphics[width=0.6\columnwidth]{photo}
	\vspace{-7cm}
\end{figure}

\begin{flushright}\small

\end{flushright}\normalsize
\framebreak


% Right frame
%%%%%%%%%%%%%%%%%%%%
\Huge\bfseries {{\color{Cyan} Ilya} {\color{Black} Chesalin}} \\
\Large\bfseries Student, programmer, hacker \\

\normalsize\normalfont

% About me
I was born in Ryazan, Russia, and I have been passionate about information
technologies for most of my life.

My main interests are Programming and Information Security. I am always looking
forward to learn new technologies to make my life easier and to create the
better world around me.

\Sep

% Skills
\CVSection{Skills}

\CVItem{Languages I speak}
\begin{multicols}{3}
\begin{compactitem}[\color{Cyan}$\circ$]
    \item Russian, native
    \item English, B2
\end{compactitem}
\end{multicols}

\SmallSep

\CVItem{Platforms}
\begin{multicols}{3}
\begin{compactitem}[\color{Cyan}$\circ$]
	\item Windows
	\item Arch Linux
	\item Debian GNU/Linux
	\item Kali Linux
	\item \ldots
\end{compactitem}
\end{multicols}

\SmallSep

\CVItem{Programming and other Languages}
\begin{multicols}{3}
\begin{compactitem}[\color{Cyan}$\circ$]
    \item Python
	\item C
    \item C++
    \item Bash/Zsh
    \item HTML/CSS
    \item JavaScript
    \item TypeScript
    \item SQL
    \item \LaTeX
	\item \ldots
\end{compactitem}
\end{multicols}

\SmallSep

\CVItem{Technologies}
\begin{multicols}{3}
\begin{compactitem}[\color{Cyan}$\circ$]
    \item Open Source \heart\
    \item Git
    \item VS Code
    \item GDB
    \item Valgrind
    \item Jenkins
    \item Qt Framework
    \item NodeJS
    \item VueJS
    \item Quasar Framework
    \item Flask
    \item PostgreSQL
    \item \ldots
\end{compactitem}
\end{multicols}

\SmallSep

\CVItem{Cybersecurity}
\begin{multicols}{3}
\begin{compactitem}[\color{Cyan}$\circ$]
    \item Radare2
    \item GDB
    \item Nmap
    \item Maltego
    \item \ldots
\end{compactitem}
\end{multicols}

\Sep

% Experience
\CVSection{Experience}

\CVItem{October 2019 - present} \\
at \textit{D-Link, Russia}
as \textit{full-stack software engineer}
\SmallSep

In October, I moved to another project, where I do full-stack web development.
\SmallSep

\texttt{Linux / JavaScript / NodeJS / VueJS / Quasar / MongoDB}
\SmallSep

\CVItem{November 2018 - September 2019} \\
at \textit{D-Link, Russia}
as \textit{system engineer}
\SmallSep

After taking cources on "Linux Embedded Programming Fundamentals" in 2017-2018, I got opportunity to join D-Link Developer team.
\SmallSep

\texttt{Linux / C / GDB / Valgrind}
\SmallSep

\CVItem{July 2017 - August 2017, July 2018 - August 2018} \\
at \textit{Svyaztransneft, JSC, Oka Region Industrial Communication Directorate}
as \textit{2nd grade techician}
\SmallSep

In the Summer of 2017, during study break, I had the opportunity to work part-time. In the Summer of 2018 I was invited back.

\clearpage
\framebreak
\framebreak

% Education
\CVSection{Education}
\CVItem{2016 - present} \\
at \textit{Ryazan State Radio Engineering University after V. Utkin}
\SmallSep

Since 2016, I have been a student at RSREU, at the departament of Information Security of the Faculty of Computer Engineering.

My speciality is Information Security of Automated Systems.

\SmallSep

\CVItem{2013 - 2016} \\
at \textit{Correspondence School of Physics and Technology of The Moscow Institute of Physics and Technology}
\SmallSep

Since ninth grade, I studied on programs of continuing education of a scientific and technical orientation.

\SmallSep

\CVItem{2005 - 2016} \\
at \textit{School 39 - Center of physical and mathematical eduation}
\SmallSep

At school I was passionate about mathematics and physics, taking part in lots of olympiads by Rosatom, MEPhI, URFO and many others.

\Sep

% Hackathons
\CVSection{Hackathons}

\CVItem{WebPurple Codenjoy} \\
by \textit{Webpurple, Ryazan}
on \textit{14 July 2018}
\SmallSep

My first ever hackathon was in Ryazan, and it was just 4 hours.

The aim was to write a bot to play tetris. Even though it was my first encounter with JavaScript, I managed to take 4th place.
\SmallSep

\texttt{JavaScript / Codenjoy}

\SmallSep

\CVItem{Hack.Moscow v3.0} \\
by \textit{Russian Hackers, Russia}
on \textit{25-27 October 2019}
\SmallSep

This was my first big hackathon, lasting 36 hours.

Our team decided to take Data Visualization track by Transparency International. We came up with an idea that helps citizens to know their officials in form of a game, that we called Chinder. We could not quite present a polished product, but we had a lot of fun and gained a lot of experience.

\texttt{Java / Android Studio / Figma}

\Sep

% Contacts
\CVSection{Contacts}

\CVItem{Email} \\
\url{evilyach@protonmail.com}

\CVItem{Github} \\
\url{github.com/evilyach}

\SmallSep